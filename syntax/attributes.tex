%
% 	tufte-latex handout for MultiMarkdown
%		http://code.google.com/p/tufte-latex/
%
%	Creates a basic handout emulating part of Edward Tufte's style
%	from some of his books
%
%	* Only h1 and h2 are valid
%	* \citep may be better than \cite
%	* \autoref doesn't work properly, may get better results with \ref
%	* footnotes don't work inside of tables
%


\documentclass{tufte-handout}
%\documentclass[justified]{tufte-handout}


% Use default packages for memoir setup

\usepackage{fancyvrb}			% Allow \verbatim et al. in footnotes
\usepackage{graphicx}			% To enable including graphics in pdf's
\usepackage{booktabs}			% Better tables
\usepackage{tabulary}			% Support longer table cells
\usepackage[utf8]{inputenc}		% For UTF-8 support
\usepackage[T1]{fontenc}		% Use T1 font encoding for accented characters
\usepackage{xcolor}				% Allow for color (annotations)
\usepackage{listings}			% Allow for source code highlighting
\usepackage{subscript}
\usepackage[normalem]{ulem}		% Support strikethrough

% CriticMarkup Support
\usepackage{soul}
\usepackage{xargs}
\usepackage{todonotes}
\newcommandx{\cmnote}[2][1=]{\linespread{1.0}\todo[linecolor=red,backgroundcolor=red!25,bordercolor=red,#1]{#2}}

% Use \ul instead of \underline since we are using soul
\let\underline\ul


\usepackage[acronym]{glossaries}
\glstoctrue
\makeglossaries
\makeindex

\renewcommand\allcapsspacing[1]{{\addfontfeature{LetterSpace=15}#1}}  
\renewcommand\smallcapsspacing[1]{{\addfontfeature{LetterSpace=10}#1}}

% Configure default metadata to avoid errors
%
%	Configure default metadata in case it's missing to avoid errors
%

\def\myauthor{Author}
\def\defaultemail{}
\def\defaultposition{}
\def\defaultdepartment{}
\def\defaultaddress{}
\def\defaultphone{}
\def\defaultfax{}
\def\defaultweb{}
\def\defaultaffiliation{}

\def\mytitle{Title}
\def\subtitle{}
\def\mykeywords{}


\def\bibliostyle{plain}
% \def\bibliocommand{}

\def\myrecipient{}

% Overwrite with your own if desired
%\input{ftp-metadata}




\def\mytitle{Link and Image Attributes}
\def\myauthor{Fletcher T. Penney}
\def\revised{2018-06-30}
\newacronym{MMD}{MMD}{MultiMarkdown}

\newacronym{MD}{MD}{Markdown}

%
%	Configure information from metadata for use in title
%

\ifx\latexauthor\undefined
\else
	\def\myauthor{\latexauthor}
\fi

\ifx\subtitle\undefined
\else
%	\addtodef{\mytitle}{}{ \\ \subtitle}
	\expandafter\def\expandafter\mytitle\expandafter{\mytitle \\ \subtitle}
\fi

\ifx\affiliation\undefined
\else
%	\addtodef{\myauthor}{}{ \\ \affiliation}
	\expandafter\def\expandafter\myauthor\expandafter{\myauthor \\ \affiliation}
\fi

\ifx\address\undefined
\else
%	\addtodef{\myauthor}{}{ \\ \address}
	\expandafter\def\expandafter\myauthor\expandafter{\myauthor \\ \address}
\fi

\ifx\phone\undefined
\else
%	\addtodef{\myauthor}{}{ \\ \phone}
	\expandafter\def\expandafter\myauthor\expandafter{\myauthor \\ \phone}
\fi

\ifx\email\undefined
\else
%	\addtodef{\myauthor}{}{ \\ \email}
	\expandafter\def\expandafter\myauthor\expandafter{\myauthor \\ \email}
\fi

\ifx\event\undefined
\else
	\date[\mydate]{\today}
\fi

\ifx\latextitle\undefined
	\def\latextitle{\mytitle}
\else
\fi


\title{\latextitle}
\author{\myauthor}

\ifx\mydate\undefined
\else
	\date{\mydate}
\fi

\begin{document}
\maketitle



\tableofcontents

\section{Link and Image Attributes}
\label{linkandimageattributes}

Adding attributes to links and images has been requested for a long time on
the Markdown discussion list. I was fairly opposed to this, as most of the
proposals really disrupted the readability of the syntax. I consider myself a
``Markdown purist'', meaning that I took John's introduction to heart:

\begin{quote}
The overriding design goal for Markdown's formatting syntax is to make
it as readable as possible. The idea is that a Markdown-formatted
document should be publishable as-is, as plain text, without looking
like it's been marked up with tags or formatting instructions. While
Markdown's syntax has been influenced by several existing text-to-HTML
filters, the single biggest source of inspiration for Markdown's
syntax is the format of plain text email.
\end{quote}

Because there was not a syntax proposal that I felt fit this goal, I was generally opposed to the idea.

Then, Choan C. Gálvez \href{http://six.pairlist.net/pipermail/markdown-discuss/2005-October/001578.html}{proposed}\footnote{\href{http://six.pairlist.net/pipermail/markdown-discuss/2005-October/001578.html}{http:\slash{}\slash{}six.pairlist.net\slash{}pipermail\slash{}markdown-discuss\slash{}2005-October\slash{}001578.html}} a brilliantly simple syntax that
stayed out of the way. By simply appending the attributes to the link
reference information, which is already removed from the text itself, it
doesn't disturb the readability.

For example:

\begin{verbatim}
This is a formatted ![image][] and a [link][] with attributes.

[image]: http://path.to/image "Image title" width=40px height=400px
[link]:  http://path.to/link.html "Some Link" class=external
         style="border: solid black 1px;"
\end{verbatim}

This will generate width and height attributes for the image, and a border
around the link. And while it can be argued that it does look ``like it's been
marked up with tags {[and]} formatting instructions'', even I can't argue too
strongly against it. The link and the title in quotes already look like some
form of markup, and the the additional tags are hardly that intrusive, and
they offer a great deal of functionality. They might even be useful in further
functions (citations?).

The attributes must continue after the other link\slash{}image data, and may contain
newlines, but must start at the beginning of the line. The format is
\texttt{attribute=value} or \texttt{attribute="multi word value"}. Currently, MultiMarkdown
does not attempt to interpret or make any use of any of these attributes.
Also, you can't have a multiword attribute span a newline.

\textbf{NOTE:} As of version 6, MultiMarkdown will also allow attributes in inline links as well:

\begin{verbatim}
Colored [link](http://example.net "Title" style="color:red")
\end{verbatim}

\input{mmd6-tufte-handout-footer}
\end{document}
