%
% 	tufte-latex handout for MultiMarkdown
%		http://code.google.com/p/tufte-latex/
%
%	Creates a basic handout emulating part of Edward Tufte's style
%	from some of his books
%
%	* Only h1 and h2 are valid
%	* \citep may be better than \cite
%	* \autoref doesn't work properly, may get better results with \ref
%	* footnotes don't work inside of tables
%


\documentclass{tufte-handout}
%\documentclass[justified]{tufte-handout}


% Use default packages for memoir setup

\usepackage{fancyvrb}			% Allow \verbatim et al. in footnotes
\usepackage{graphicx}			% To enable including graphics in pdf's
\usepackage{booktabs}			% Better tables
\usepackage{tabulary}			% Support longer table cells
\usepackage[utf8]{inputenc}		% For UTF-8 support
\usepackage[T1]{fontenc}		% Use T1 font encoding for accented characters
\usepackage{xcolor}				% Allow for color (annotations)
\usepackage{listings}			% Allow for source code highlighting
\usepackage{subscript}
\usepackage[normalem]{ulem}		% Support strikethrough

% CriticMarkup Support
\usepackage{soul}
\usepackage{xargs}
\usepackage{todonotes}
\newcommandx{\cmnote}[2][1=]{\linespread{1.0}\todo[linecolor=red,backgroundcolor=red!25,bordercolor=red,#1]{#2}}

% Use \ul instead of \underline since we are using soul
\let\underline\ul


\usepackage[acronym]{glossaries}
\glstoctrue
\makeglossaries
\makeindex

\renewcommand\allcapsspacing[1]{{\addfontfeature{LetterSpace=15}#1}}  
\renewcommand\smallcapsspacing[1]{{\addfontfeature{LetterSpace=10}#1}}

% Configure default metadata to avoid errors
%
%	Configure default metadata in case it's missing to avoid errors
%

\def\myauthor{Author}
\def\defaultemail{}
\def\defaultposition{}
\def\defaultdepartment{}
\def\defaultaddress{}
\def\defaultphone{}
\def\defaultfax{}
\def\defaultweb{}
\def\defaultaffiliation{}

\def\mytitle{Title}
\def\subtitle{}
\def\mykeywords{}


\def\bibliostyle{plain}
% \def\bibliocommand{}

\def\myrecipient{}

% Overwrite with your own if desired
%\input{ftp-metadata}




\def\mytitle{Fenced Code Blocks}
\def\myauthor{Fletcher T. Penney}
\def\revised{2018-06-30}
\newacronym{MMD}{MMD}{MultiMarkdown}

\newacronym{MD}{MD}{Markdown}

%
%	Configure information from metadata for use in title
%

\ifx\latexauthor\undefined
\else
	\def\myauthor{\latexauthor}
\fi

\ifx\subtitle\undefined
\else
%	\addtodef{\mytitle}{}{ \\ \subtitle}
	\expandafter\def\expandafter\mytitle\expandafter{\mytitle \\ \subtitle}
\fi

\ifx\affiliation\undefined
\else
%	\addtodef{\myauthor}{}{ \\ \affiliation}
	\expandafter\def\expandafter\myauthor\expandafter{\myauthor \\ \affiliation}
\fi

\ifx\address\undefined
\else
%	\addtodef{\myauthor}{}{ \\ \address}
	\expandafter\def\expandafter\myauthor\expandafter{\myauthor \\ \address}
\fi

\ifx\phone\undefined
\else
%	\addtodef{\myauthor}{}{ \\ \phone}
	\expandafter\def\expandafter\myauthor\expandafter{\myauthor \\ \phone}
\fi

\ifx\email\undefined
\else
%	\addtodef{\myauthor}{}{ \\ \email}
	\expandafter\def\expandafter\myauthor\expandafter{\myauthor \\ \email}
\fi

\ifx\event\undefined
\else
	\date[\mydate]{\today}
\fi

\ifx\latextitle\undefined
	\def\latextitle{\mytitle}
\else
\fi


\title{\latextitle}
\author{\myauthor}

\ifx\mydate\undefined
\else
	\date{\mydate}
\fi

\begin{document}
\maketitle



\tableofcontents

\section{Fenced Code Blocks}
\label{fencedcodeblocks}

In addition to the regular indented code block that Markdown uses, you can use ``fenced'' code blocks in MultiMarkdown. These code blocks do not have to be indented, and can also be configured to be compatible with a third party syntax highlighter. These code blocks should begin with 3 to 5 backticks, an optional language specifier (if using a syntax highlighter), and should end with the same number of backticks you started with:

\begin{verbatim}
```perl
# Demonstrate Syntax Highlighting if you link to highlight.js #
# http://softwaremaniacs.org/soft/highlight/en/
print "Hello, world!\n";
$a = 0;
while ($a < 10) {
print "$a...\n";
$a++;
}
```
\end{verbatim}

\begin{lstlisting}[language=perl]
# Demonstrate Syntax Highlighting if you link to highlight.js #
# http://softwaremaniacs.org/soft/highlight/en/
print "Hello, world!\n";
$a = 0;
while ($a < 10) {
print "$a...\n";
$a++;
}
\end{lstlisting}

I don't recommend any specific syntax highlighter, but have used the following metadata to set things up. It may or may not work for you:

\begin{verbatim}
html header:	<link rel="stylesheet" href="http://yandex.st/highlightjs/7.3/styles/default.min.css">
	<script src="http://yandex.st/highlightjs/7.3/highlight.min.js"></script>
	<script>hljs.initHighlightingOnLoad();</script>
\end{verbatim}

Fenced code blocks are particularly useful when including another file (\href{https://fletcher.github.io/MultiMarkdown-6/syntax/transclusion.html}{File Transclusion}\footnote{\href{https://fletcher.github.io/MultiMarkdown-6/syntax/transclusion.html}{https:\slash \slash fletcher.github.io\slash MultiMarkdown-6\slash syntax\slash transclusion.html}}), and you want to show the \emph{source} of the file, not what the file looks like when processed by MultiMarkdown.

\textbf{NOTE}: In MultiMarkdown v6, if there is no closing fence, then the code block continues until the end of the document.

\input{mmd6-tufte-handout-footer}
\end{document}
