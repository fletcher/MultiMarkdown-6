%
% 	tufte-latex handout for MultiMarkdown
%		http://code.google.com/p/tufte-latex/
%
%	Creates a basic handout emulating part of Edward Tufte's style
%	from some of his books
%
%	* Only h1 and h2 are valid
%	* \citep may be better than \cite
%	* \autoref doesn't work properly, may get better results with \ref
%	* footnotes don't work inside of tables
%


\documentclass{tufte-handout}
%\documentclass[justified]{tufte-handout}


% Use default packages for memoir setup

\usepackage{fancyvrb}			% Allow \verbatim et al. in footnotes
\usepackage{graphicx}			% To enable including graphics in pdf's
\usepackage{booktabs}			% Better tables
\usepackage{tabulary}			% Support longer table cells
\usepackage[utf8]{inputenc}		% For UTF-8 support
\usepackage[T1]{fontenc}		% Use T1 font encoding for accented characters
\usepackage{xcolor}				% Allow for color (annotations)
\usepackage{listings}			% Allow for source code highlighting
\usepackage{subscript}
\usepackage[normalem]{ulem}		% Support strikethrough

% CriticMarkup Support
\usepackage{soul}
\usepackage{xargs}
\usepackage{todonotes}
\newcommandx{\cmnote}[2][1=]{\linespread{1.0}\todo[linecolor=red,backgroundcolor=red!25,bordercolor=red,#1]{#2}}

% Use \ul instead of \underline since we are using soul
\let\underline\ul


\usepackage[acronym]{glossaries}
\glstoctrue
\makeglossaries
\makeindex

\renewcommand\allcapsspacing[1]{{\addfontfeature{LetterSpace=15}#1}}  
\renewcommand\smallcapsspacing[1]{{\addfontfeature{LetterSpace=10}#1}}

% Configure default metadata to avoid errors
%
%	Configure default metadata in case it's missing to avoid errors
%

\def\myauthor{Author}
\def\defaultemail{}
\def\defaultposition{}
\def\defaultdepartment{}
\def\defaultaddress{}
\def\defaultphone{}
\def\defaultfax{}
\def\defaultweb{}
\def\defaultaffiliation{}

\def\mytitle{Title}
\def\subtitle{}
\def\mykeywords{}


\def\bibliostyle{plain}
% \def\bibliocommand{}

\def\myrecipient{}

% Overwrite with your own if desired
%\input{ftp-metadata}




\def\mytitle{Images}
\def\myauthor{Fletcher T. Penney}
\def\revised{2018-06-30}
\newacronym{MMD}{MMD}{MultiMarkdown}

\newacronym{MD}{MD}{Markdown}

%
%	Configure information from metadata for use in title
%

\ifx\latexauthor\undefined
\else
	\def\myauthor{\latexauthor}
\fi

\ifx\subtitle\undefined
\else
%	\addtodef{\mytitle}{}{ \\ \subtitle}
	\expandafter\def\expandafter\mytitle\expandafter{\mytitle \\ \subtitle}
\fi

\ifx\affiliation\undefined
\else
%	\addtodef{\myauthor}{}{ \\ \affiliation}
	\expandafter\def\expandafter\myauthor\expandafter{\myauthor \\ \affiliation}
\fi

\ifx\address\undefined
\else
%	\addtodef{\myauthor}{}{ \\ \address}
	\expandafter\def\expandafter\myauthor\expandafter{\myauthor \\ \address}
\fi

\ifx\phone\undefined
\else
%	\addtodef{\myauthor}{}{ \\ \phone}
	\expandafter\def\expandafter\myauthor\expandafter{\myauthor \\ \phone}
\fi

\ifx\email\undefined
\else
%	\addtodef{\myauthor}{}{ \\ \email}
	\expandafter\def\expandafter\myauthor\expandafter{\myauthor \\ \email}
\fi

\ifx\event\undefined
\else
	\date[\mydate]{\today}
\fi

\ifx\latextitle\undefined
	\def\latextitle{\mytitle}
\else
\fi


\title{\latextitle}
\author{\myauthor}

\ifx\mydate\undefined
\else
	\date{\mydate}
\fi

\begin{document}
\maketitle



\tableofcontents

\section{Images }
\label{images}

The basic syntax for images in Markdown is:

\begin{verbatim}
![Alt text](/path/to/img.jpg)

![Alt text](/path/to/img.jpg "Optional title")


![Alt text][id]

[id]: url/to/image  "Optional title attribute"
\end{verbatim}

In addition to the attributes you can use with links and images (described in \href{https://fletcher.github.io/MultiMarkdown-6/syntax/attributes.html}{another section}\footnote{\href{https://fletcher.github.io/MultiMarkdown-6/syntax/attributes.html}{https:\slash \slash fletcher.github.io\slash MultiMarkdown-6\slash syntax\slash attributes.html}}), MultiMarkdown also adds a few additional features. If an image is the only thing in a paragraph, it is treated as a block level element:

\begin{verbatim}
This image (![Alt text](/path/to/img.jpg))
is different than the following image:

![Alt text](/path/to/img.jpg)
\end{verbatim}

The resulting HTML is:

\begin{verbatim}
<p>This image (<img src="/path/to/img.jpg" alt="Alt text" />)
is different than the following image:</p>

<figure>
<img src="/path/to/img.jpg" alt="Alt text" />
<figcaption>Alt text</figcaption>
</figure>
\end{verbatim}

The first one would be an inline image. The second one (in HTML) would be wrapped in an HTML \texttt{figure} element. In this case, the \texttt{alt} text is also used as a figure caption, and can contain MultiMarkdown syntax (e.g. bold, emph, etc.). The alt text is not specifically designed to limit which MultiMarkdown is supported, but there will be limits and block level elements aren't supported.

\input{mmd6-tufte-handout-footer}
\end{document}
